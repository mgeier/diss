\chapter{Conclusion and Future Work}
\label{sec:conclusion}

The previous chapters have shown
the definition of the Audio Scene Description Format (ASDF),
the implementation of a library for loading ASDF files
and its integration into different software for spatial audio reproduction.
All involved programs and libraries are available as open-source software
and for free.
Everyone can use the ASDF to create spatial audio scenes
and the ASDF library and its integrations can be used to play them back.
The software implementation can also be used as a basis for further
experimentation and to prototype features that are currently not implemented.

There has been no systematic evaluation of the format yet,
but everyone is encouraged to try it out and make their own judgement.
As chapter~\ref{sec:existing-formats} has shown,
there are currently no dedicated authoring formats
for spatial audio scenes available.
Therefore, no direct comparison is possible.
From the example scenes in appendix~\ref{sec:asdf} it should be apparent
that the ASDF syntax is more concise and easier to hand-write
than any of the other text-based formats
shown in chapter~\ref{sec:existing-formats}.
The feature set of the ASDF, however,
is very much limited compared to some of the other formats that were mentioned.
The scope of the format is intentionally chosen to be very narrow.
It is focused on the description of
movements and rotation of scene objects over time.
This description is based on different types of \emph{splines},
which are covered in considerable depth in appendix~\ref{sec:splines}.
Nearly everything else is out of scope,
as section~\ref{sec:out-of-scope} describes.

The implementation of the ASDF library covers all features of the format
as currently defined,
but of course additional functionality could be implemented.
One example would be the recording of movements using a tracking system,
followed by a data reduction by means of some kind of \emph{curve fitting},
which would allow to create a high-level declarative description
from a low-level stream of sampled data.

The three-dimensional GUI prototype shown in section~\ref{sec:visualization}
could of course be improved in many ways.
It would be a massive endeavour,
but maybe an application could be implemented that allows
creating and editing trajectories graphically, including the possibility
to animate nested local coordinate systems.
This would also need some kind of elaborate timeline editor
to be able to define the relationships between objects in time.
It is unlikely that such an extensive GUI project would be started
(and more importantly, would lead to a usable program).
More realistically,
some features of the ASDF could probably be improved in order to
simplify the scene authoring process via editing plain text files.
The ASDF has its own issue tracker\footnote{\url{%
https://github.com/AudioSceneDescriptionFormat/asdf/issues}}
where future features can be discussed.

Appendix~\ref{sec:splines} contains a thorough write-up about Euclidean splines
and rotation splines, but it is of course by far not exhaustive,
and more material
-- including more types of splines --
can be added in the future.
Some new findings might even lead to changes in future versions of the ASDF.
For example,
more and better end conditions could be brought forward,
which could replace the end conditions that are currently used in the ASDF.
There is still a lot of old and new literature that has not been incorporated.
Many aspects that might be worth considering in the future
are already mentioned in the issue tracker
for the \emph{splines} project\footnote{\url{%
https://github.com/AudioSceneDescriptionFormat/splines/issues}}.

The ASDF allows the creation of position trajectories and of orientation
trajectories and it can even handle a combination of both.
However, when a scene object or a group of objects moves
along a position trajectory, it does not automatically change its orientation
according to the curvature of the trajectory.
This is a feature that might be desirable for scene authors.

The splines that are currently used in the ASDF
guarantee a continuous change of velocity between spline segments,
but the acceleration vector
-- \ie the second derivative --
is allowed to be discontinuous.
It might be interesting to investigate whether
choosing a different type of spline that guarantees continuity of acceleration
will have any noticeable advantages.

Since all the specifications, software and documentation is publicly available,
it should be easy for anyone who is interested to build upon this work.
