Nach einem kurzen Abriss über die Geschichte
der \emph{räumlichen Audiowiedergabe}
wird das Konzept der \emph{objektbasierten} Audiowiedergabe erklärt
und die Notwendigkeit von \emph{räumlichen Audioszenen} wird festgestellt.
Einige existierende Beschreibungsformate für
objektbasierte Audioszenen werden betrachtet,
mit Hauptaugenmerk auf die Beschreibung der Bewegung von Szenenobjektbewegungen
im Zeitverlauf.
Ein neues Format für Szenenautoren namens
Audio Scene Description Format (ASDF) wird präsentiert.
Seine Beschreibung der Szenenobjektbewegungen fußt auf
mehreren Arten von \emph{Splines},
die gründlich untersucht werden,
sowohl für Position als auch für Rotation.
Zu guter Letzt wird die Implementierung
einer quelloffenen ASDF Softwarebibliothek sowie zwei Einbindungen dieser
Bibliothek präsentiert,
die es ab sofort jedem ermöglichen, das ASDF auszuprobieren.

% vim: spelllang=de

