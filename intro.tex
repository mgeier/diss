\renewcommand{\dictumwidth}{0.5\textwidth}
\setchapterpreamble[ol][0.5\textwidth]{%
\dictum[\cite{garity1941fantasound}]{%
It has been found that
by artificially causing the source of sound
to move rapidly in space
the result can be highly dramatic and desirable.}}

\addchap{Introduction}

The goal of \emph{spatial audio reproduction}
is to create auditory events for a listener or multiple listeners,
which are perceived as arriving from specific spatial directions.
This can be achieved either with headphones or with arrangements
of multiple loudspeakers.
Such technology can be used,
for example,
for creating
spatial music performances,
audio plays
or spatial sound tracks for movies.

To set everything into perspective, chapter~\ref{sec:history}
will give an overview about the historic development of spatial audio
reproduction and present some old and new techniques
for creating spatial audio experiences for an audience.
This is also where the term \emph{channel-based} will be introduced
to classify reproduction systems
where each loudspeaker driving signal
is stored and distributed in its own separate channel.
This will be contrasted with
the term \emph{object-based},
which represents a whole new reproduction paradigm.
Instead of delivering an \emph{audio mix} which is tied to a fixed loudspeaker
setup (or to two headphone channels),
a so-called \emph{audio scene} comprised of individual sound sources is
created which holds all source signals as well as additional data like source
positions and other scene parameters.
Based on such an audio scene,
the loudspeaker or headphone signals are generated in real time
for any given reproduction setup.

Chapter~\ref{sec:existing-formats}
will describe some existing file formats to store object-based
audio scenes.
The main focus will be on the handling of three-dimensional movements
of scene objects.
Most of those formats were not explicitly designed for authoring,
and even though many formats are text-based,
their syntaxes make it unnecessarily hard to author scenes by hand.
Most formats do not provide any
high-level \emph{declarative}\footnote{%
For an explanation of the term \emph{declarative} see
section~\ref{sec:declarative-procedural-sampled}.}
syntax for the description of smooth movements.

The Audio Scene Description Format (ASDF)
-- a new authoring format for object-based audio scenes --
will be presented in chapter~\ref{sec:format-development}.
Instead of using spatial relationships as the main structural element,
it uses temporal relationships.
Elements of a scene can happen in parallel or in sequence.
By nesting so-called \emph{time containers},
arbitrary time relationships can be defined.
Both audio clips and the \emph{transforms} that control their spatial
positions and movements
are part of the same timeline.
Spatial relations are still very important,
but they are not defined by the top-level file structure.
Instead, spatial relationships have to be explicitly established
by referencing other scene objects or transforms via their IDs.

A detailed description of all features of the ASDF
is available in appendix~\ref{sec:asdf},
which also includes a lot of small examples.
The \emph{declarative}
definition of movements of objects in a scene
-- including the rotation of objects and groups of objects --
is based on \emph{splines}.
An extensive review of all the relevant types of splines
is provided in appendix~\ref{sec:splines}.
The appendices are an integral part of this thesis
and they should not be overlooked.
The reason why those two appendices are not part of the main text
is that they are actually self-contained projects,
which are separately available online.
Furthermore,
they are meant to evolve beyond being used as part of this thesis.
Appendix~\ref{sec:splines} not only illustrates
the properties of all the types of splines used in the ASDF,
but it also provides tools for investigating further types
and maybe developing new types.
It thoroughly derives the fundamentals of interpolatory polynomial splines
in Euclidean space and it applies the same methods
-- as far as possible -- to rotation splines.
Three-dimensional rotation splines are notoriously hard to visualize
and since this thesis is printed on paper,
only sequences of snapshots of rotations can be shown.
However,
in the online
HTML version\footnote{\url{https://splines.readthedocs.io/}},
a number of animations are available that can much better
illustrate the behavior of subtly different rotation splines.

It is an important goal of this thesis
to not only define a theoretical file format on paper,
but also to enable its practical usage
in order to be able to properly evaluate its capabilities and weaknesses.
Therefore, an open-source software library has been implemented,
which is described in chapter~\ref{sec:implementation}.
This library has also been integrated in a stand-alone
software for spatial audio reproduction,
which means that the ASDF is ready to be tried out by anyone who is interested.


\addsec{Acknowledgements}

I would like to thank Sebastian Möller,
head of the Quality and Usability (QU) Lab,
Technische Universität Berlin\footnote{\url{https://www.tu.berlin/qu/}},
for taking me in as a PhD student
and for giving me all the support I could wish for
during the time I worked there.
I am indebted to Sascha Spors,
at the time head of the spatial audio group at QU Lab and later professor
at the Institute of Communications Engineering at Universität
Rostock\footnote{\url{https://www.int.uni-rostock.de/}},
for acting as my thesis advisor since the beginning
and for his support over all those years.
Thanks to Jens Ahrens
for introducing me to the QU Lab in the first place
and for his continuous collaboration on the \emph{SoundScape Renderer}
which now serves as a proving ground for the ASDF.
Thanks to all my colleagues at QU Lab in Berlin
and at the Institute of Communications Engineering in Rostock
for many fruitful discussions
and for contributing to a great working experience.
I would like to thank Jürgen Herre
for helping me find the motivation to finally finish this thesis.
A big thank you goes to
Tracy Harris for proofreading the whole text including the appendices.
Thanks to Mikhail Korotiaev for several suggestions
for improvements in appendix~\ref{sec:splines}.
Finally, I would like to thank all the
maintainers and contributors of the many open-source software projects
that were used for creating this thesis and the associated software and
documentation: \LaTeX, Sphinx, Jupyter, Matplotlib, NumPy, SymPy,
Python and Rust, just to name a few.
Specifically, I would like to thank
Jean-François B.\footnote{\url{https://github.com/jfbu}}
for his work on the \LaTeX\ output of
\emph{nbsphinx}\footnote{\url{https://nbsphinx.readthedocs.io/}},
without which the Jupyter code cells in the appendix would look much
less pleasing.
